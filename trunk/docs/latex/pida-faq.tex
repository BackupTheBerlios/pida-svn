\documentclass[10pt,a4paper,english]{article}
\usepackage{babel}
\usepackage{ae}
\usepackage{aeguill}
\usepackage{shortvrb}
\usepackage[latin1]{inputenc}
\usepackage{tabularx}
\usepackage{longtable}
\setlength{\extrarowheight}{2pt}
\usepackage{amsmath}
\usepackage{graphicx}
\usepackage{color}
\usepackage{multirow}
\usepackage{ifthen}
\usepackage[colorlinks=true,linkcolor=blue,urlcolor=blue]{hyperref}
\usepackage[DIV12]{typearea}
%% generator Docutils: http://docutils.sourceforge.net/
\newlength{\admonitionwidth}
\setlength{\admonitionwidth}{0.9\textwidth}
\newlength{\docinfowidth}
\setlength{\docinfowidth}{0.9\textwidth}
\newlength{\locallinewidth}
\newcommand{\optionlistlabel}[1]{\bf #1 \hfill}
\newenvironment{optionlist}[1]
{\begin{list}{}
  {\setlength{\labelwidth}{#1}
   \setlength{\rightmargin}{1cm}
   \setlength{\leftmargin}{\rightmargin}
   \addtolength{\leftmargin}{\labelwidth}
   \addtolength{\leftmargin}{\labelsep}
   \renewcommand{\makelabel}{\optionlistlabel}}
}{\end{list}}
\newlength{\lineblockindentation}
\setlength{\lineblockindentation}{2.5em}
\newenvironment{lineblock}[1]
{\begin{list}{}
  {\setlength{\partopsep}{\parskip}
   \addtolength{\partopsep}{\baselineskip}
   \topsep0pt\itemsep0.15\baselineskip\parsep0pt
   \leftmargin#1}
 \raggedright}
{\end{list}}
% begin: floats for footnotes tweaking.
\setlength{\floatsep}{0.5em}
\setlength{\textfloatsep}{\fill}
\addtolength{\textfloatsep}{3em}
\renewcommand{\textfraction}{0.5}
\renewcommand{\topfraction}{0.5}
\renewcommand{\bottomfraction}{0.5}
\setcounter{totalnumber}{50}
\setcounter{topnumber}{50}
\setcounter{bottomnumber}{50}
% end floats for footnotes
% some commands, that could be overwritten in the style file.
\newcommand{\rubric}[1]{\subsection*{~\hfill {\it #1} \hfill ~}}
\newcommand{\titlereference}[1]{\textsl{#1}}
% end of "some commands"
\title{The PIDA FAQ}
\author{}
\date{}
\hypersetup{
pdftitle={The PIDA FAQ},
pdfauthor={Ali Afshar}
}
\raggedbottom
\begin{document}
\maketitle

%___________________________________________________________________________
\begin{center}
\begin{tabularx}{\docinfowidth}{lX}
\textbf{Author}: &
	Ali Afshar \\
\textbf{Contact}: &
	\href{mailto:aafshar@gmail.com}{aafshar@gmail.com} \\
\end{tabularx}
\end{center}

\setlength{\locallinewidth}{\linewidth}


%___________________________________________________________________________

\hypertarget{general}{}
\pdfbookmark[0]{General}{general}
\section*{General}


%___________________________________________________________________________

\hypertarget{is-pida-available-for-windows}{}
\pdfbookmark[1]{Is PIDA available for Windows?}{is-pida-available-for-windows}
\subsection*{Is PIDA available for Windows?}


%___________________________________________________________________________

\hypertarget{vim}{}
\pdfbookmark[0]{Vim}{vim}
\section*{Vim}


%___________________________________________________________________________

\hypertarget{i-close-buffers-in-vim-why-are-they-still-visible-in-pida}{}
\pdfbookmark[1]{I close buffers in Vim. Why are they still visible in PIDA?}{i-close-buffers-in-vim-why-are-they-still-visible-in-pida}
\subsection*{I close buffers in Vim. Why are they still visible in PIDA?}

If you have closed a vim buffer using \texttt{:bd} or \texttt{Delete} from the
\texttt{Buffer} menu, the buffer will still remain in the PIDA buffer list. This is
not a bug but a feature of how Vim works. It does not actually delete the
buffer on \texttt{:bd} but merely hides it.

The solution is to use \texttt{:bw}. This properly unloads the buffer and PIDA
correctly recognises this situation. If you have read the Vim docs (like a
good Vimmer) you know hat the documentation regarding \texttt{:bw} is a bit scary,
and asks you to \emph{make sure you know what you are doing}. Rest assured, it
works fine.

\end{document}
