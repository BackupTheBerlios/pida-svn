\documentclass[10pt,a4paper,english]{article}
\usepackage{babel}
\usepackage{ae}
\usepackage{aeguill}
\usepackage{shortvrb}
\usepackage[latin1]{inputenc}
\usepackage{tabularx}
\usepackage{longtable}
\setlength{\extrarowheight}{2pt}
\usepackage{amsmath}
\usepackage{graphicx}
\usepackage{color}
\usepackage{multirow}
\usepackage{ifthen}
\usepackage[colorlinks=true,linkcolor=blue,urlcolor=blue]{hyperref}
\usepackage[DIV12]{typearea}
%% generator Docutils: http://docutils.sourceforge.net/
\newlength{\admonitionwidth}
\setlength{\admonitionwidth}{0.9\textwidth}
\newlength{\docinfowidth}
\setlength{\docinfowidth}{0.9\textwidth}
\newlength{\locallinewidth}
\newcommand{\optionlistlabel}[1]{\bf #1 \hfill}
\newenvironment{optionlist}[1]
{\begin{list}{}
  {\setlength{\labelwidth}{#1}
   \setlength{\rightmargin}{1cm}
   \setlength{\leftmargin}{\rightmargin}
   \addtolength{\leftmargin}{\labelwidth}
   \addtolength{\leftmargin}{\labelsep}
   \renewcommand{\makelabel}{\optionlistlabel}}
}{\end{list}}
\newlength{\lineblockindentation}
\setlength{\lineblockindentation}{2.5em}
\newenvironment{lineblock}[1]
{\begin{list}{}
  {\setlength{\partopsep}{\parskip}
   \addtolength{\partopsep}{\baselineskip}
   \topsep0pt\itemsep0.15\baselineskip\parsep0pt
   \leftmargin#1}
 \raggedright}
{\end{list}}
% begin: floats for footnotes tweaking.
\setlength{\floatsep}{0.5em}
\setlength{\textfloatsep}{\fill}
\addtolength{\textfloatsep}{3em}
\renewcommand{\textfraction}{0.5}
\renewcommand{\topfraction}{0.5}
\renewcommand{\bottomfraction}{0.5}
\setcounter{totalnumber}{50}
\setcounter{topnumber}{50}
\setcounter{bottomnumber}{50}
% end floats for footnotes
% some commands, that could be overwritten in the style file.
\newcommand{\rubric}[1]{\subsection*{~\hfill {\it #1} \hfill ~}}
\newcommand{\titlereference}[1]{\textsl{#1}}
% end of "some commands"
\title{The PIDA Manual}
\author{}
\date{}
\hypersetup{
pdftitle={The PIDA Manual},
pdfauthor={Ali Afshar}
}
\raggedbottom
\begin{document}
\maketitle

%___________________________________________________________________________
\begin{center}
\begin{tabularx}{\docinfowidth}{lX}
\textbf{Author}: &
	Ali Afshar \\
\textbf{Contact}: &
	\href{mailto:aafshar@gmail.com}{aafshar@gmail.com} \\
\end{tabularx}
\end{center}

\setlength{\locallinewidth}{\linewidth}
\hypertarget{table-of-contents}{}
\pdfbookmark[0]{Table Of Contents}{table-of-contents}
\subsubsection*{~\hfill Table Of Contents\hfill ~}
\begin{list}{}{}
\item {} \href{\#front-matter}{Front Matter}
\begin{list}{}{}
\item {} \href{\#introduction}{Introduction}

\item {} \href{\#copyright}{Copyright}

\item {} \href{\#authors}{Authors}

\item {} \href{\#contributors}{Contributors}

\end{list}

\item {} \href{\#getting-started}{Getting Started}
\begin{list}{}{}
\item {} \href{\#requirements-to-run-pida}{Requirements to run PIDA}
\begin{list}{}{}
\item {} \href{\#minimum-requirements}{Minimum Requirements}

\item {} \href{\#optional-requirements}{Optional Requirements}

\end{list}

\item {} \href{\#installing-pida}{Installing PIDA}

\item {} \href{\#running-pida}{Running PIDA}
\begin{list}{}{}
\item {} \href{\#running-pida-after-installation}{Running pida after installation}

\item {} \href{\#running-pida-without-installing}{Running pida without installing}

\end{list}

\end{list}

\item {} \href{\#using-pida}{Using PIDA}
\begin{list}{}{}
\item {} \href{\#projects}{Projects}
\begin{list}{}{}
\item {} \href{\#adding-a-project-to-the-workbench}{Adding a project to the workbench}

\item {} \href{\#using-a-project}{Using a project}

\item {} \href{\#configuring-a-project}{Configuring a project}

\end{list}

\item {} \href{\#version-control}{Version Control}
\begin{list}{}{}
\item {} \href{\#project-management-version-control}{Project Management Version Control}

\item {} \href{\#file-browsing-version-control}{File Browsing Version Control}

\end{list}

\item {} \href{\#terminal-emulators}{Terminal Emulators}

\end{list}

\item {} \href{\#developing-pida}{Developing PIDA}

\item {} \href{\#appendix-a}{Appendix A}

\end{list}



%___________________________________________________________________________

\hypertarget{front-matter}{}
\pdfbookmark[0]{Front Matter}{front-matter}
\section*{Front Matter}


%___________________________________________________________________________

\hypertarget{introduction}{}
\pdfbookmark[1]{Introduction}{introduction}
\subsection*{Introduction}

PIDA is an integrated development environment for all types of development. It
is written in Python using the PyGTK Toolkit.

PIDA is different from other IDEs. Rather than attempting to write a set of
development tools of its own, PIDA uses tools that the developer has available.
In this regards PIDA is a framework for assembling a bespoke IDE. PIDA allows
you to choose the editor you wish to use (yes, Vim out of the box works).

Although still a young application, pIDA can already boast a huge number of
features because of the power of some of the tools it integrates. For example
features such as code completion and syntax highlighting are well implemented in
PIDA's integrated editors far better than any editor built for a commercial
IDE.

Additionally PIDA insists on stealing excellent ideas from applications it
cannot embed. For example the Rapid Application Development in the style of
Microsoft's development products is achieved by the combination of \href{http://gazpacho.sicem.biz/}{Gazpacho} (a
user interface designer) and {\color{red}\bfseries{}Tepache{\_}} (a code sketcher), via the text editor.


%___________________________________________________________________________

\hypertarget{copyright}{}
\pdfbookmark[1]{Copyright}{copyright}
\subsection*{Copyright}

PIDA is released under the {\color{red}\bfseries{}MIT{\_}} license. This is not a particularly philosophical
decision, except that the PIDA developers consider it a good thing that PIDA is
not \href{http://www.opensource.org/licenses/gpl-license.php}{GPL}, or even closed source.

The PIDA Project is owned by Ali Afshar (this author).


%___________________________________________________________________________

\hypertarget{authors}{}
\pdfbookmark[1]{Authors}{authors}
\subsection*{Authors}
\begin{itemize}
\item {} 
Ali Afshar

\item {} 
Bernard Pratz

\item {} 
Alejandro Mery

\end{itemize}


%___________________________________________________________________________

\hypertarget{contributors}{}
\pdfbookmark[1]{Contributors}{contributors}
\subsection*{Contributors}

Stephen Holmes - A consistent and competent source of pain and suffering.


%___________________________________________________________________________

\hypertarget{getting-started}{}
\pdfbookmark[0]{Getting Started}{getting-started}
\section*{Getting Started}


%___________________________________________________________________________

\hypertarget{requirements-to-run-pida}{}
\pdfbookmark[1]{Requirements to run PIDA}{requirements-to-run-pida}
\subsection*{Requirements to run PIDA}

Your requirements will largely depend on what you want to do with PIDA.


%___________________________________________________________________________

\hypertarget{minimum-requirements}{}
\pdfbookmark[2]{Minimum Requirements}{minimum-requirements}
\subsubsection*{Minimum Requirements}
\begin{description}
%[visit_definition_list_item]
\item[Python] (\textbf{2.4})
%[visit_definition]

Python is most likely installed on UNIX systems. PIDA requires version 2.4,
which is available to download at {\color{red}\bfseries{}py24download{\_}}.

%[depart_definition]
%[depart_definition_list_item]
%[visit_definition_list_item]
\item[PyGTK] (\textbf{2.6})
%[visit_definition]

GTK is the graphical toolkit used in PIDA, and PyGTK is the set of bindings
for the Python programing language. You should use the latest version
available at {\color{red}\bfseries{}pygtkdownload{\_}}

%[depart_definition]
%[depart_definition_list_item]
\end{description}


%___________________________________________________________________________

\hypertarget{optional-requirements}{}
\pdfbookmark[2]{Optional Requirements}{optional-requirements}
\subsubsection*{Optional Requirements}
\begin{description}
%[visit_definition_list_item]
\item[Gazpacho User Interface Designer] (\textbf{0.6.4})
%[visit_definition]

Gazpacho is a GTK (Glade cmpatible) user interface designer. The latest
version is available at the \href{http://gazpacho.sicem.biz/}{Gazpacho} web site.

%[depart_definition]
%[depart_definition_list_item]
\end{description}


%___________________________________________________________________________

\hypertarget{installing-pida}{}
\pdfbookmark[1]{Installing PIDA}{installing-pida}
\subsection*{Installing PIDA}

If you can get PIDA from your distribution, this is best. Otherwise you will
need to download the source tarball. Unpack the tarball, and in the top-level
directory, issue the command (you may require super user access on your computer for a system install):
\begin{quote}{\ttfamily \raggedright \noindent
python~setup.py~install
}\end{quote}

PIDA will now be installed in your default python location, and be available
to all users of the system.
\begin{center}\begin{sffamily}
\fbox{\parbox{\admonitionwidth}{
\textbf{\large Note}
\vspace{2mm}

If you do not wish to install PIDA, it can be run from the local directory.
(See \href{\#running-pida-without-installing}{Running pida without installing})
}}
\end{sffamily}
\end{center}


%___________________________________________________________________________

\hypertarget{running-pida}{}
\pdfbookmark[1]{Running PIDA}{running-pida}
\subsection*{Running PIDA}


%___________________________________________________________________________

\hypertarget{running-pida-after-installation}{}
\pdfbookmark[2]{Running pida after installation}{running-pida-after-installation}
\subsubsection*{Running pida after installation}

If PIDA has been installed, simply issue the command:
\begin{quote}{\ttfamily \raggedright \noindent
pida
}\end{quote}

If correctly installed, PIDA will start.


%___________________________________________________________________________

\hypertarget{running-pida-without-installing}{}
\pdfbookmark[2]{Running pida without installing}{running-pida-without-installing}
\subsubsection*{Running pida without installing}

The \texttt{develop.sh} script in the top-level source directory can be used to run
PIDA without installing system-wide. To execute it, issue the command:
\begin{quote}{\ttfamily \raggedright \noindent
./develop.sh
}\end{quote}

The script generates a PIDA egg in a temporary directory for the duration of
the session.


%___________________________________________________________________________

\hypertarget{using-pida}{}
\pdfbookmark[0]{Using PIDA}{using-pida}
\section*{Using PIDA}

PIDA is very varied in its features and what you may want to do with it might
not be what someone else might want to do with it (this is fine). In order to
familiarise yourself with PIDA, the following chapters are designed to take
you through the basic common functionality that we think you would all like to
use.


%___________________________________________________________________________

\hypertarget{projects}{}
\pdfbookmark[1]{Projects}{projects}
\subsection*{Projects}

PIDA projects are the way in which PIDA organises a set of files. The default
project type maps to a single source directory, which is then used for quick
navigation and version control functions.


%___________________________________________________________________________

\hypertarget{adding-a-project-to-the-workbench}{}
\pdfbookmark[2]{Adding a project to the workbench}{adding-a-project-to-the-workbench}
\subsubsection*{Adding a project to the workbench}

Firstly, from the \emph{Project} menu select \emph{Add Project}, and Enter the
information into the newly displayed form.
\begin{description}
%[visit_definition_list_item]
\item[Name] %[visit_definition]

The name you would like to use for the project

%[depart_definition]
%[depart_definition_list_item]
%[visit_definition_list_item]
\item[Save In] %[visit_definition]

The directory you would like to save the project file in (or the default
pida projects directory by default).

%[depart_definition]
%[depart_definition_list_item]
%[visit_definition_list_item]
\item[Type] %[visit_definition]

The type of project this project is

%[depart_definition]
%[depart_definition_list_item]
\end{description}

Once you have entered this information click \emph{ok}.

You will be presented with the initial project configuration dialog for the
project.

Depending on the type of project, you will have different options. The most
common option is \emph{Source Directory}. This is the directory that will be
navigated to when clicking on a project, and the directory that is used for
project functions, including version control. When you are happy with the
configuration, press the \emph{save} button.

Your new project will have appeared on the project list, and is available to
browse and use.
\begin{center}\begin{sffamily}
\fbox{\parbox{\admonitionwidth}{
\textbf{\large Note}
\vspace{2mm}

The project file may be stored in the project source directory if
required. The initial value of the project source directory actually
defaults to the location of the project source file. This allows you to add
the project file to a version control system and monitor the changes.
}}
\end{sffamily}
\end{center}


%___________________________________________________________________________

\hypertarget{using-a-project}{}
\pdfbookmark[2]{Using a project}{using-a-project}
\subsubsection*{Using a project}

First, Locate the project list. It is in the pane marked \emph{plugins} and
has an icon signifying a project. This pane will be used to access projects.

Selecting a project from this project list will open a file manager in the
source directory of the project, whatever that is configured to be.

Right-clicking on a project gives the context menu. This context menu is
divided into three sections of contexts.
\begin{description}
%[visit_definition_list_item]
\item[Directory ] %[visit_definition]

These are file system actions to be performed on the source directory.

%[depart_definition]
%[depart_definition_list_item]
%[visit_definition_list_item]
\item[Source Code] %[visit_definition]

These are version control commands to be performed in the context of the
project.

%[depart_definition]
%[depart_definition_list_item]
%[visit_definition_list_item]
\item[Project] %[visit_definition]

These are actions to be performed on the actual project object, e.g.
project configuration.

%[depart_definition]
%[depart_definition_list_item]
\end{description}


%___________________________________________________________________________

\hypertarget{configuring-a-project}{}
\pdfbookmark[2]{Configuring a project}{configuring-a-project}
\subsubsection*{Configuring a project}

Projects are configured using the project configuration dialog. To open the
project configuration dialog, either:
\newcounter{listcnt1}
\begin{list}{\arabic{listcnt1}.}
{
\usecounter{listcnt1}
\setlength{\rightmargin}{\leftmargin}
}
\item {} 
Select \emph{Properties} from the \emph{Project} menu.

\item {} 
Right click on a project, and in the context menu, select \emph{Configure this
project} from the \emph{Project} submenu.

\end{list}

You should click the \emph{Save} button when you have finished and are happy.

The \emph{Undo} button allows you to revert changes to the configuration back to
the last saved state.

The \emph{Cancel} button closes the dialog without saving any changes. Closing the
dialog manually will have the same effect as pressing \emph{Cancel}.


%___________________________________________________________________________

\hypertarget{version-control}{}
\pdfbookmark[1]{Version Control}{version-control}
\subsection*{Version Control}

PIDA automatically detects which version control system you are using for a
particular source directory. This allows you to choose the version control
system you wish to use.

PIDA currently supports:
\begin{itemize}
\item {} 
CVS

\item {} 
Subversion

\item {} 
Darcs

\item {} 
Mercurial

\item {} 
Monotone

\item {} 
Bzr

\item {} 
Arch

\end{itemize}

Version control is used throughout PIDA in 3 ways which are outlined below.


%___________________________________________________________________________

\hypertarget{project-management-version-control}{}
\pdfbookmark[2]{Project Management Version Control}{project-management-version-control}
\subsubsection*{Project Management Version Control}

The project list states the version control system for a project. When a
project is selected, main version control commands (from the main menu and
main toolbar) will be executed in the source directory of the project,
automatically using the correct version control system.

The version control commands may also be accessed using the context menu
made available by right-clicking on a project.


%___________________________________________________________________________

\hypertarget{file-browsing-version-control}{}
\pdfbookmark[2]{File Browsing Version Control}{file-browsing-version-control}
\subsubsection*{File Browsing Version Control}

The built-in file browser autoimatically lists version control information
for listed files. This information appears as a standard set of letters
(e.g. \emph{M} for a locally modified file) adjacent to filenames in the browser.

To use this, click on any project, in order to open the browser at the
project's source directory.

When right clicking on a file or directory, you are given a list of version
control commands which can be carried out n the file or directory.


%___________________________________________________________________________

\hypertarget{terminal-emulators}{}
\pdfbookmark[1]{Terminal Emulators}{terminal-emulators}
\subsection*{Terminal Emulators}


%___________________________________________________________________________

\hypertarget{developing-pida}{}
\pdfbookmark[0]{Developing PIDA}{developing-pida}
\section*{Developing PIDA}


%___________________________________________________________________________

\hypertarget{appendix-a}{}
\pdfbookmark[0]{Appendix A}{appendix-a}
\section*{Appendix A}

The MIT License:
\begin{quote}{\ttfamily \raggedright \noindent
Copyright~(c)~2005-2006~The~PIDA~Project~\\
~\\
Permission~is~hereby~granted,~free~of~charge,~to~any~person~obtaining~a~copy~of~\\
this~software~and~associated~documentation~files~(the~"Software"),~to~deal~in~\\
the~Software~without~restriction,~including~without~limitation~the~rights~to~\\
use,~copy,~modify,~merge,~publish,~distribute,~sublicense,~and/or~sell~copies~of~\\
the~Software,~and~to~permit~persons~to~whom~the~Software~is~furnished~to~do~so,~\\
subject~to~the~following~conditions:~\\
~\\
The~above~copyright~notice~and~this~permission~notice~shall~be~included~in~all~\\
copies~or~substantial~portions~of~the~Software.~\\
~\\
THE~SOFTWARE~IS~PROVIDED~"AS~IS",~WITHOUT~WARRANTY~OF~ANY~KIND,~EXPRESS~OR~\\
IMPLIED,~INCLUDING~BUT~NOT~LIMITED~TO~THE~WARRANTIES~OF~MERCHANTABILITY,~FITNESS~\\
FOR~A~PARTICULAR~PURPOSE~AND~NONINFRINGEMENT.~IN~NO~EVENT~SHALL~THE~AUTHORS~OR~\\
COPYRIGHT~HOLDERS~BE~LIABLE~FOR~ANY~CLAIM,~DAMAGES~OR~OTHER~LIABILITY,~WHETHER~\\
IN~AN~ACTION~OF~CONTRACT,~TORT,~TURTLE,~OR~OTHERWISE,~ARISING~FROM,~OUT~OF~OR~IN~\\
CONNECTION~WITH~THE~SOFTWARE~OR~THE~USE~OR~OTHER~DEALINGS~IN~THE~SOFTWARE.
}\end{quote}


%___________________________________________________________________________

\section*{Docutils System Messages}

Unknown target name: ``tepache''.


Unknown target name: ``mit''.


Unknown target name: ``py24download''.


Unknown target name: ``pygtkdownload''.


\end{document}
