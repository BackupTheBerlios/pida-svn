% Complete documentation on the extended LaTeX markup used for Python
% documentation is available in ``Documenting Python'', which is part
% of the standard documentation for Python.  It may be found online
% at:
%
%     http://www.python.org/doc/current/doc/doc.html

\documentclass{manual}

\title{The PIDA Manual}

\author{Ali Afshar}

% Please at least include a long-lived email address;
% the rest is at your discretion.
\authoraddress{
	Email: \email{aafshar@gmail.com}
}

\date{January 14, 2006}		% update before release!
				% Use an explicit date so that reformatting
				% doesn't cause a new date to be used.  Setting
				% the date to \today can be used during draft
				% stages to make it easier to handle versions.

\release{1.0}			% release version; this is used to define the
				% \version macro

\makeindex			% tell \index to actually write the .idx file
\makemodindex			% If this contains a lot of module sections.


\begin{document}

\maketitle

% This makes the contents more accessible from the front page of the HTML.
\ifhtml
\chapter*{Front Matter\label{front}}
\fi

%\input{copyright}

\begin{abstract}

\noindent
PIDA is an integrated development environment.

\end{abstract}

\tableofcontents


\chapter{...}

Installing PIDA


\appendix
\chapter{...}

My appendix.

The \code{\e appendix} markup need not be repeated for additional
appendices.


\end{document}
