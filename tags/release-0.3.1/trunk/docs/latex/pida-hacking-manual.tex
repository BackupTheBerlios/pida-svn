\documentclass[10pt,a4paper,english]{article}
\usepackage{babel}
\usepackage{ae}
\usepackage{aeguill}
\usepackage{shortvrb}
\usepackage[latin1]{inputenc}
\usepackage{tabularx}
\usepackage{longtable}
\setlength{\extrarowheight}{2pt}
\usepackage{amsmath}
\usepackage{graphicx}
\usepackage{color}
\usepackage{multirow}
\usepackage{ifthen}
\usepackage[colorlinks=true,linkcolor=blue,urlcolor=blue]{hyperref}
\usepackage[DIV12]{typearea}
%% generator Docutils: http://docutils.sourceforge.net/
\newlength{\admonitionwidth}
\setlength{\admonitionwidth}{0.9\textwidth}
\newlength{\docinfowidth}
\setlength{\docinfowidth}{0.9\textwidth}
\newlength{\locallinewidth}
\newcommand{\optionlistlabel}[1]{\bf #1 \hfill}
\newenvironment{optionlist}[1]
{\begin{list}{}
  {\setlength{\labelwidth}{#1}
   \setlength{\rightmargin}{1cm}
   \setlength{\leftmargin}{\rightmargin}
   \addtolength{\leftmargin}{\labelwidth}
   \addtolength{\leftmargin}{\labelsep}
   \renewcommand{\makelabel}{\optionlistlabel}}
}{\end{list}}
\newlength{\lineblockindentation}
\setlength{\lineblockindentation}{2.5em}
\newenvironment{lineblock}[1]
{\begin{list}{}
  {\setlength{\partopsep}{\parskip}
   \addtolength{\partopsep}{\baselineskip}
   \topsep0pt\itemsep0.15\baselineskip\parsep0pt
   \leftmargin#1}
 \raggedright}
{\end{list}}
% begin: floats for footnotes tweaking.
\setlength{\floatsep}{0.5em}
\setlength{\textfloatsep}{\fill}
\addtolength{\textfloatsep}{3em}
\renewcommand{\textfraction}{0.5}
\renewcommand{\topfraction}{0.5}
\renewcommand{\bottomfraction}{0.5}
\setcounter{totalnumber}{50}
\setcounter{topnumber}{50}
\setcounter{bottomnumber}{50}
% end floats for footnotes
% some commands, that could be overwritten in the style file.
\newcommand{\rubric}[1]{\subsection*{~\hfill {\it #1} \hfill ~}}
\newcommand{\titlereference}[1]{\textsl{#1}}
% end of "some commands"
\title{The PIDA's Hacking Manual}
\author{}
\date{}
\hypersetup{
pdftitle={The PIDA's Hacking Manual},
pdfauthor={Tiago Cogumbreiro;Ali Afshar}
}
\raggedbottom
\begin{document}
\maketitle

%___________________________________________________________________________
\begin{center}
\begin{tabularx}{\docinfowidth}{lX}
\textbf{Author}: &
	Tiago Cogumbreiro \\
\textbf{Author}: &
	Ali Afshar \\
\textbf{Contact}: &
	\href{mailto:cogumbreiro@users.sf.net}{cogumbreiro@users.sf.net} \\
\end{tabularx}
\end{center}

\setlength{\locallinewidth}{\linewidth}
\hypertarget{table-of-contents}{}
\pdfbookmark[0]{Table Of Contents}{table-of-contents}
\subsubsection*{~\hfill Table Of Contents\hfill ~}
\begin{list}{}{}
\item {} \href{\#creating-gtk-action-s}{Creating gtk.Action's}
\begin{list}{}{}
\item {} \href{\#the-hard-way}{The Hard Way}

\item {} \href{\#the-fast-way}{The Fast Way}

\end{list}

\item {} \href{\#integrating-into-menu-and-toolbar}{Integrating into menu and toolbar}

\item {} \href{\#using-icons}{Using Icons}

\end{list}



%___________________________________________________________________________

\hypertarget{creating-gtk-action-s}{}
\pdfbookmark[0]{Creating gtk.Action's}{creating-gtk-action-s}
\section*{Creating gtk.Action's}

First you need to know that there's a gtk.Action group associated with
each service. It's name is the 'classname'.
It can be accessed through 'self.action{\_}group'.


%___________________________________________________________________________

\hypertarget{the-hard-way}{}
\pdfbookmark[1]{The Hard Way}{the-hard-way}
\subsection*{The Hard Way}

In order to add actions to the service's action group you just need
to add them in the 'init()' method which is called to initialized the
service.

Example:
\begin{quote}{\ttfamily \raggedright \noindent
import~pida.core.service~as~service~\\
import~gtk~\\
~\\
class~MyService(service.service):~\\
~~~~def~init(self):~\\
~~~~~~~~self.action{\_}group.add{\_}actions({[}~\\
~~~~~~~~~~~~("GoForward",gtk.STOCK{\_}GO{\_}FORWARD,~None,~\\
~~~~~~~~~~~~~"This~makes~it~go~forward"),~\\
~~~~~~~~])~\\
~~~~~~~~~\\
~~~~~~~~act~=~self.action{\_}group.get{\_}action("GoForward")~\\
~~~~~~~~act.connect("activate",~self.on{\_}go{\_}forward)~\\
~~~~~\\
~~~~def~on{\_}go{\_}forward(self,~action):~\\
~~~~~~~~print~"Go~forward!"
}\end{quote}


%___________________________________________________________________________

\hypertarget{the-fast-way}{}
\pdfbookmark[1]{The Fast Way}{the-fast-way}
\subsection*{The Fast Way}

Another way of creating actions is the \emph{implicit creation}. Pida has a nice
feature that turns every method you prefix with a '{\color{red}\bfseries{}act{\_}}' into a 'gtk.Action' and
then the action name.

The action name is generated with the 'module+servicename+action{\_}name', where
'action{\_}name' is the method name ignoring the '{\color{red}\bfseries{}act{\_}}' part. The tooltip text is
generated from the doc string of the method.

So, a simillar example would be:
\begin{quote}{\ttfamily \raggedright \noindent
import~pida.core.service~as~service~\\
~\\
class~MyService(service.service):~\\
~~~~def~act{\_}go{\_}forward(self,~action):~\\
~~~~~~~~"{}"{}"This~makes~it~go~forward"{}"{}"~\\
~~~~~~~~print~"Go~forward!"
}\end{quote}

As you can see the code is way smaller but it has the following limitations:
\begin{quote}
\begin{itemize}
\item {} 
implicitly connects to the 'activate' signal

\item {} 
the stock icon is fetched from the first word (separated by '{\_}'), so in the
case of gtk.STOCK{\_}GO{\_}FORWARD there's a no go.

\item {} 
it can only create gtk.Action's, not gtk.ToggleAction's nor
gtk.RadioAction's.

\end{itemize}
\end{quote}


%___________________________________________________________________________

\hypertarget{integrating-into-menu-and-toolbar}{}
\pdfbookmark[0]{Integrating into menu and toolbar}{integrating-into-menu-and-toolbar}
\section*{Integrating into menu and toolbar}

In order to put your gtk.Action's in toolbar and menus you have to
have a concept of gtk.UIManager. You have 5 available placeholders to put your
actions for your toolbar:
\begin{quote}
\begin{itemize}
\item {} 
'OpenFileToolbar'

\item {} 
'SaveFileToolbar'

\item {} 
'EditToolbar'

\item {} 
'ProjectToolbar'

\item {} 
'VcToolbar'

\item {} 
'ToolsToolbar'

\end{itemize}
\end{quote}
\begin{description}
%[visit_definition_list_item]
\item[And 4 available placeholders for your menu entries:] %[visit_definition]
\begin{itemize}
\item {} 
'OpenFileMenu'

\item {} 
'SaveFileMenu'

\item {} 
'ExtrasFileMenu'

\item {} 
'GlobalFileMenu'

\end{itemize}

%[depart_definition]
%[depart_definition_list_item]
\end{description}

This is also the order of which they were created.

The next part is that you need to
return a UIManager definition with the following nature: implement a method
named 'get{\_}menu{\_}definition' in your service class:
\begin{quote}{\ttfamily \raggedright \noindent
import~pida.core.service~as~service~\\
~\\
class~MyService(service.service):~\\
~\\
~~~~def~get{\_}menu{\_}definition(self):~\\
~~~~~~~~return~"{}"{}"~\\
~~~~~~~~~~~~<menubar>~\\
~~~~~~~~~~~~~~~~<placeholder~name="ExtrasFileMenu">~\\
~~~~~~~~~~~~~~~~~~~~<menu~name="my{\_}menu{\_}entry"~action="MyAction">~\\
~~~~~~~~~~~~~~~~</placeholder>~\\
~~~~~~~~~~~~</menubar>~\\
~~~~~~~~~~~~<toolbar>~\\
~~~~~~~~~~~~~~~~<placeholder~name="ToolsToolbar">~\\
~~~~~~~~~~~~~~~~~~~~<toolitem~name="my{\_}tool{\_}item"~action="MyAction"~/>~\\
~~~~~~~~~~~~~~~~</placeholder>~\\
~~~~~~~~~~~~</toolbar>~\\
~~~~~~~~"{}"{}"
}\end{quote}

In this example we've plugged our action 'MyAction' to the menu and the toolbar.


%___________________________________________________________________________

\hypertarget{using-icons}{}
\pdfbookmark[0]{Using Icons}{using-icons}
\section*{Using Icons}

All icons will be in SVG format, and will be placed in data/icons. To use them
they can be got using:
\begin{quote}{\ttfamily \raggedright \noindent
pidagtk.icons.icons.get(name)
}\end{quote}

or using name as a stock{\_}id, where name is the name of the icon without the svg
extension.


%___________________________________________________________________________

\section*{Docutils System Messages}

Unknown target name: ``act''.


Unknown target name: ``act''.


\end{document}
